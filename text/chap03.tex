\chapter{User Documentation}
\label{chap:user}

This chapter describes how the suggester should be used. It is divided into 2 parts:
\begin{itemize}
    \item User guide \ref{user_guide} – usage described from the end user point of view.
    \item Administrator guide \ref{administrator_guide} – describes how to modify the suggester configuration.
\end{itemize}

\section{User Guide}
\label{user_guide}
When the OpenGrok main web page is opened the user sees something similar to the Figure \ref{opengrok_main}.
When the user types text into one of the search fields, suggestions are shown as can be seen in Figure \ref{suggestions_pic}.

\begin{figure}[htbp]
    \centering
    \includegraphics[width=145mm]{../img/suggestions_pic.png}
    \caption{Suggestions example}
    \label{suggestions_pic}
\end{figure}

The numbers in the Figure \ref{suggestions_pic} represent:
\begin{enumerate}
    \item Typed value into the search field. In this case letter \textit{t}.
    \item List of suggestions.
    \item Time it took to generate the suggestions. (Optional.)
    \item In which projects the term was found. (Optional.)
    \item Score of the terms. (Optional.)
    \item Projects for which suggestions are shown.
\end{enumerate}

\subsection{Selecting the Suggestions}
The suggestions can be navigated by using:
\begin {itemize}
    \item \textbf{Keyboard} – the suggestion can be focused by using up $\uparrow$ and down $\downarrow$ arrow keys.
    The suggestion can be selected by pressing the \textit{enter} (carriage return) or \textit{tab} key.
    The suggestions window can also be closed by pressing the \textit{esc} key.
    \item \textbf{Mouse} – by hovering over the suggestion it is focused. By clicking on the suggestion it
    is selected.
\end{itemize}

\subsubsection{Focused Suggestion}
If a suggestion is focused, then its background color is changed to \textit{blue}. In case of keyboard focus,
the text of the field shows the result value as if this suggestion was selected.
It looks like in the Figure \ref{suggestion_focused}.

\begin{figure}[htbp]
    \centering
    \includegraphics[width=145mm]{../img/suggestions_focused.png}
    \caption{Focused suggestion}
    \label{suggestion_focused}
\end{figure}

The numbers in the Figure \ref{suggestion_focused} represent:
\begin{enumerate}
    \item The field already contains the value \textit{package} even though only the prefix \textit{p} is actually written.
    \item Focused suggestion.
\end{enumerate}

\subsubsection{Errors}
It might be a common occurrence that users type an invalid query. For instance, \textit{)} is missing at the end of
the query \textit{(test}. Therefore, the user is notified if the Lucene cannot parse the query or some other
error occurs. This case is pictured in the Figure \ref{suggestions_error}.

\begin{figure}[htbp]
    \centering
    \includegraphics[width=145mm]{../img/suggestions_error.png}
    \caption{Suggestions error}
    \label{suggestions_error}
\end{figure}

The numbers in the Figure \ref{suggestions_error} represent:
\begin{enumerate}
    \item Error notification.
    \item Detail of the error. This is only shown if the user hovers over the error notification.
\end{enumerate}

\subsubsection{Minisearch}
While traversing a source code or history, a minisearch field is available in the navigation menu. It allows users
to search in the current project. The suggestions are enabled for it as well. The result can look like in the Figure
\ref{suggestions_minisearch}.

\begin{figure}[htbp]
    \centering
    \includegraphics[width=145mm]{../img/minisearch.png}
    \caption{Suggestions for minisearch}
    \label{suggestions_minisearch}
\end{figure}

In the Figure \ref{suggestions_minisearch} the numbers represent:
\begin{enumerate}
    \item Checkbox that specifies if the search (and consequently suggestions) should be restricted to the directory
    in which the displayed file is located. In this example it adds the following value to the \textit{path} field:
\begin{code}
"/protobuf/objectivec/ProtocolBuffers_iOS.xcodeproj/xcshareddata/
xcschemes/"
\end{code}
    \item Search button that launches the search.
    \item Suggestions window as described before.
\end{enumerate}

\subsubsection{Timeout}
As specified in \ref{restricting_load}, the system may be under load and it might return only a partial result or no
suggestions at all. In the case of the partial result, the message \textit{"Partial result due to timeout"} will be
displayed at the bottom of the suggestions window similarly as in the Figure \ref{partial_result}.
\begin{figure}[htbp]
    \centering
    \includegraphics[width=100mm]{../img/partial_result.png}
    \caption{Message if only partial result was returned}
    \label{partial_result}
\end{figure}

If no suggestions were returned, then
it may signify two cases:
\begin{itemize}
    \item The query yields no results. In that case a message \textit{"No matches found"} will be displayed as in the
    Figure \ref{no_matches}.
    \begin{figure}[htbp]
        \centering
        \includegraphics[width=100mm]{../img/no_matches.png}
        \caption{Message if no matches were found for the specified query}
        \label{no_matches}
    \end{figure}

    \item The system timed out and did not find any possible suggestions. In that case no suggestions window will be displayed.
\end{itemize}

\subsubsection{Support for Lucene's Field Modifier}
Lucene queries can contain a field modifier which specifies the field for which the query should be matched (syntax \textit{\{field\}:\{query\}}). OpenGrok
actually performs this modification when the user types queries into different search inputs. However, it is also possible
to use these modifiers in the \textit{Full Search} input field. For example, it is possible to write a query
\texttt{defs:a$\vert$} (\texttt{$\vert$} represents a caret position) and the suggester will suggest terms with prefix
\texttt{a} exactly as it would if the text \texttt{a|} were written into \textit{Definition} input.

\section{Administrator Guide}
\label{administrator_guide}

There are multiple ways to modify the suggester configuration described in \ref{suggester_config}:
\begin{itemize}
    \item By modifying the OpenGrok's XML configuration described in \ref{opengrok_configuration}. An example:
\begin{code}
<java version="1.8.0_162" class="java.beans.XMLDecoder">
  <object
      class="org.opensolaris.opengrok.configuration.Configuration"
      id="Configuration0">
    <void property="suggesterConfig">
      <void property="allowComplexQueries">
        <boolean>false</boolean>
      </void>
      <void property="allowMostPopular">
        <boolean>false</boolean>
      </void>
      <void property="allowedFields">
        <object class="java.util.Collections" method="singleton">
          <string>defs</string>
        </object>
      </void>
      <void property="allowedProjects">
        <object class="java.util.Collections" method="singleton">
          <string>fruitonserver</string>
        </object>
      </void>
      <void property="enabled">
        <boolean>false</boolean>
      </void>
      <void property="maxProjects">
        <int>2</int>
      </void>
      <void property="maxResults">
        <int>11</int>
      </void>
      <void property="minChars">
        <int>2</int>
      </void>
      <void property="rebuildCronConfig">
        <string>5 * * * *</string>
      </void>
      <void property="showProjects">
        <boolean>false</boolean>
      </void>
      <void property="showScores">
        <boolean>true</boolean>
      </void>
      <void property="showTime">
        <boolean>true</boolean>
      </void>
      <void property="buildTerminationTime">
        <int>500</int>
      </void>
    </void>
  </object>
</java>
\end{code}
    \item By using the REST API configuration endpoint. It is possible to send a JSON representation of the suggester
    configuration. Only the properties which differ from the default configuration need to be specified. Example data:
\begin{code}
{"enabled": "false"}
\end{code}
    Suggester configuration endpoint:
\begin{code}
/api/v1/configuration/suggesterConfig
\end{code}

\end{itemize}
